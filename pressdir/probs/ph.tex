\documentstyle[note]{article}
\pagestyle{empty}
\begin{document}
\vspace*{2.15in}
\begin{center}
\begin{large}
{\bf PRESS}: {\bf Pr}olog {\bf E}quation {\bf S}olving {\bf S}ystem
\end{large}
\end{center}
PRESS was, at the time it was written, the first really large-scale
program written in Prolog. The aim of the PRESS project was not so much
to build a powerful equation-solving system (although it would get
an `A' at `A' Level Mathematics!) but to test out the ideas which
people in the Artificial Intelligence Department at Edinburgh had
about how to do theorem proving in general.
For example, how does a High School student ``know'' how to solve
\[ 	log_2 x + 4.log_x 2 = 5 \]

Here is PRESS's proof: note how ``human-like'' it is.

\begin{footnotesize}

\begin{verbatim}
This problem comes from the London 1978 A level exam.
We are asked to find the value(s) of for which
	log(2,x) + 4.log(x,2) = 5.

Solving log(2, x) + 4 * log(x, 2) = 5 for x

Rewriting equation in terms of log(2, x)
gives log(2, x) + 4 * log(2, x) ^ -1 = 5

Substituting   x1 for log(2, x) gives
 4 * x1 ^ -1 + x1 = 5

Multiply through by x1 to get 

x1 ^ 2 + -5 * x1 + 4 = 0

Using quadratic equation formula

Solutions are x1 = 4 and x1 = 1

Applying substitution 
    x1 = log(2, x)

   to    : 
    x1 = 4 # x1 = 1

   gives : 
    log(2, x) = 4 # log(2, x) = 1

Solving disjunct log(2, x) = 4

    x = 16
     (by Isolation)

Solving disjunct log(2, x) = 1

    x = 2
     (by Isolation)

Answer is : 
(X1 # X2)
  where :
    X1 =  x = 16
    X2 =  x = 2

\end{verbatim}

\end{footnotesize}

\bibliography{/usr/local/lib/tex.biblio/papers}
\bibliographystyle{harvardbib}

\end{document}
